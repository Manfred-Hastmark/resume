%-----------------------------------------------------------------------------------------------------------------------------------------------%
%	The MIT License (MIT)
%
%	Copyright (c) 2019 Jan Küster
%
%	Permission is hereby granted, free of charge, to any person obtaining a copy
%	of this software and associated documentation files (the "Software"), to deal
%	in the Software without restriction, including without limitation the rights
%	to use, copy, modify, merge, publish, distribute, sublicense, and/or sell
%	copies of the Software, and to permit persons to whom the Software is
%	furnished to do so, subject to the following conditions:
%	
%	THE SOFTWARE IS PROVIDED "AS IS", WITHOUT WARRANTY OF ANY KIND, EXPRESS OR
%	IMPLIED, INCLUDING BUT NOT LIMITED TO THE WARRANTIES OF MERCHANTABILITY,
%	FITNESS FOR A PARTICULAR PURPOSE AND NONINFRINGEMENT. IN NO EVENT SHALL THE
%	AUTHORS OR COPYRIGHT HOLDERS BE LIABLE FOR ANY CLAIM, DAMAGES OR OTHER
%	LIABILITY, WHETHER IN AN ACTION OF CONTRACT, TORT OR OTHERWISE, ARISING FROM,
%	OUT OF OR IN CONNECTION WITH THE SOFTWARE OR THE USE OR OTHER DEALINGS IN
%	THE SOFTWARE.
%	
%
%-----------------------------------------------------------------------------------------------------------------------------------------------%


%============================================================================%
%
%	DOCUMENT DEFINITION
%
%============================================================================%

%we use article class because we want to fully customize the page and don't use a cv template
\documentclass[10pt,A4]{article}	


%----------------------------------------------------------------------------------------
%	ENCODING
%----------------------------------------------------------------------------------------

% we use utf8 since we want to build from any machine
\usepackage[utf8]{inputenc}		

%----------------------------------------------------------------------------------------
%	LOGIC
%----------------------------------------------------------------------------------------

% provides \isempty test
\usepackage{xstring, xifthen}

%----------------------------------------------------------------------------------------
%	FONT BASICS
%----------------------------------------------------------------------------------------

% some tex-live fonts - choose your own

%\usepackage[defaultsans]{droidsans}
%\usepackage[default]{comfortaa}
%\usepackage{cmbright}
\usepackage[default]{raleway}
%\usepackage{fetamont}
%\usepackage[default]{gillius}
%\usepackage[light,math]{iwona}
%\usepackage[thin]{roboto} 

% set font default
\renewcommand*\familydefault{\sfdefault} 	
\usepackage[T1]{fontenc}

% more font size definitions
\usepackage{moresize}

%----------------------------------------------------------------------------------------
%	FONT AWESOME ICONS
%---------------------------------------------------------------------------------------- 

% include the fontawesome icon set
\usepackage{fontawesome}

% use to vertically center content
% credits to: http://tex.stackexchange.com/questions/7219/how-to-vertically-center-two-images-next-to-each-other
\newcommand{\vcenteredinclude}[1]{\begingroup
\setbox0=\hbox{\includegraphics{#1}}%
\parbox{\wd0}{\box0}\endgroup}

% use to vertically center content
% credits to: http://tex.stackexchange.com/questions/7219/how-to-vertically-center-two-images-next-to-each-other
\newcommand*{\vcenteredhbox}[1]{\begingroup
\setbox0=\hbox{#1}\parbox{\wd0}{\box0}\endgroup}

% icon shortcut
\newcommand{\icon}[3] { 							
	\makebox(#2, #2){\textcolor{maincol}{\csname fa#1\endcsname}}
}	

% icon with text shortcut
\newcommand{\icontext}[4]{ 						
	\vcenteredhbox{\icon{#1}{#2}{#3}}  \hspace{2pt}  \parbox{0.9\mpwidth}{\textcolor{#4}{#3}}
}

% icon with website url
\newcommand{\iconhref}[5]{ 						
    \vcenteredhbox{\icon{#1}{#2}{#5}}  \hspace{2pt} \href{#4}{\textcolor{#5}{#3}}
}

% icon with email link
\newcommand{\iconemail}[5]{ 						
    \vcenteredhbox{\icon{#1}{#2}{#5}}  \hspace{2pt} \href{mailto:#4}{\textcolor{#5}{#3}}
}

%----------------------------------------------------------------------------------------
%	PAGE LAYOUT  DEFINITIONS
%----------------------------------------------------------------------------------------

% page outer frames (debug-only)
% \usepackage{showframe}		

% we use paracol to display breakable two columns
\usepackage{paracol}

% define page styles using geometry
\usepackage[a4paper]{geometry}

% remove all possible margins
\geometry{top=1cm, bottom=1cm, left=1cm, right=1cm}

\usepackage{fancyhdr}
\pagestyle{empty}

% space between header and content
% \setlength{\headheight}{0pt}

% indentation is zero
\setlength{\parindent}{0mm}

%----------------------------------------------------------------------------------------
%	TABLE /ARRAY DEFINITIONS
%---------------------------------------------------------------------------------------- 

% extended aligning of tabular cells
\usepackage{array}

% custom column right-align with fixed width
% use like p{size} but via x{size}
\newcolumntype{x}[1]{%
>{\raggedleft\hspace{0pt}}p{#1}}%


%----------------------------------------------------------------------------------------
%	GRAPHICS DEFINITIONS
%---------------------------------------------------------------------------------------- 

%for header image
\usepackage{graphicx}

% use this for floating figures
% \usepackage{wrapfig}
% \usepackage{float}
% \floatstyle{boxed} 
% \restylefloat{figure}

%for drawing graphics		
\usepackage{tikz}				
\usetikzlibrary{shapes, backgrounds,mindmap, trees}

%----------------------------------------------------------------------------------------
%	Color DEFINITIONS
%---------------------------------------------------------------------------------------- 
\usepackage{transparent}
\usepackage{color}

% primary color
\definecolor{maincol}{RGB}{ 225, 0, 0 }

% accent color, secondary
% \definecolor{accentcol}{RGB}{ 250, 150, 10 }

% dark color
\definecolor{darkcol}{RGB}{ 70, 70, 70 }

% light color
\definecolor{lightcol}{RGB}{245,245,245}


% Package for links, must be the last package used
\usepackage[hidelinks]{hyperref}

% returns minipage width minus two times \fboxsep
% to keep padding included in width calculations
% can also be used for other boxes / environments
\newcommand{\mpwidth}{\linewidth-\fboxsep-\fboxsep}
	


%============================================================================%
%
%	CV COMMANDS
%
%============================================================================%

%----------------------------------------------------------------------------------------
%	 CV LIST
%----------------------------------------------------------------------------------------

% renders a standard latex list but abstracts away the environment definition (begin/end)
\newcommand{\cvlist}[1] {
	\begin{itemize}{#1}\end{itemize}
}

%----------------------------------------------------------------------------------------
%	 CV TEXT
%----------------------------------------------------------------------------------------

% base class to wrap any text based stuff here. Renders like a paragraph.
% Allows complex commands to be passed, too.
% param 1: *any
\newcommand{\cvtext}[1] {
	\begin{tabular*}{1\mpwidth}{p{0.98\mpwidth}}
		\parbox{1\mpwidth}{#1}
	\end{tabular*}
}

%----------------------------------------------------------------------------------------
%	CV SECTION
%----------------------------------------------------------------------------------------

% Renders a a CV section headline with a nice underline in main color.
% param 1: section title
\newcommand{\cvsection}[1] {
	\vspace{14pt}
	\cvtext{
		\textbf{\LARGE{\textcolor{darkcol}{\uppercase{#1}}}}\\[-4pt]
		\textcolor{maincol}{ \rule{0.1\textwidth}{2pt} } \\
	}
}

%----------------------------------------------------------------------------------------
%	META SKILL
%----------------------------------------------------------------------------------------

% Renders a progress-bar to indicate a certain skill in percent.
% param 1: name of the skill / tech / etc.
% param 2: level (for example in years)
% param 3: percent, values range from 0 to 1
\newcommand{\cvskill}[3] {
	\begin{tabular*}{1\mpwidth}{p{0.72\mpwidth}  r}
 		\textcolor{black}{\textbf{#1}} & \textcolor{maincol}{#2}\\
	\end{tabular*}%
	
	\hspace{4pt}
	\begin{tikzpicture}[scale=1,rounded corners=2pt,very thin]
		\fill [lightcol] (0,0) rectangle (1\mpwidth, 0.15);
		\fill [maincol] (0,0) rectangle (#3\mpwidth, 0.15);
  	\end{tikzpicture}%
}


%----------------------------------------------------------------------------------------
%	 CV EVENT
%----------------------------------------------------------------------------------------

% Renders a table and a paragraph (cvtext) wrapped in a parbox (to ensure minimum content
% is glued together when a pagebreak appears).
% Additional Information can be passed in text or list form (or other environments).
% the work you did
% param 1: time-frame i.e. Sep 14 - Jan 15 etc.
% param 2:	 event name (job position etc.)
% param 3: Customer, Employer, Industry
% param 4: Short description
% param 5: work done (optional)
% param 6: technologies include (optional)
% param 7: achievements (optional)
\newcommand{\cvevent}[7] {
	
	% we wrap this part in a parbox, so title and description are not separated on a pagebreak
	% if you need more control on page breaks, remove the parbox
	\parbox{\mpwidth}{
		\begin{tabular*}{1\mpwidth}{p{0.72\mpwidth}  r}
	 		\textcolor{black}{\textbf{#2}} & \colorbox{maincol}{\makebox[0.25\mpwidth]{\textcolor{white}{#1}}} \\
			\textcolor{maincol}{\textbf{#3}} & \\
		\end{tabular*}\\[8pt]
	
		\ifthenelse{\isempty{#4}}{}{
			\cvtext{#4}\\
		}
	}

	\ifthenelse{\isempty{#5}}{}{
		\vspace{9pt}
		{#5}
	}

	\ifthenelse{\isempty{#6}}{}{
		\vspace{9pt}
		\cvtext{\textbf{Technologies include:}}\\
		{#6}
	}

	\ifthenelse{\isempty{#7}}{}{
		\vspace{9pt}
		\cvtext{\textbf{Achievements include:}}\\
		{#7}
	}
	\vspace{14pt}
}

%----------------------------------------------------------------------------------------
%	 CV META EVENT
%----------------------------------------------------------------------------------------

% Renders a CV event on the sidebar
% param 1: title
% param 2: subtitle (optional)
% param 3: customer, employer, etc,. (optional)
% param 4: info text (optional)
\newcommand{\cvmetaevent}[4] {
	\textcolor{maincol} {\cvtext{\textbf{\begin{flushleft}#1\end{flushleft}}}}

	\ifthenelse{\isempty{#2}}{}{
	\textcolor{darkcol} {\cvtext{\textbf{#2}} }
	}

	\ifthenelse{\isempty{#3}}{}{
		\cvtext{{ \textcolor{darkcol} {#3} }}\\
	}

	\cvtext{#4}\\[14pt]
}

%---------------------------------------------------------------------------------------
%	QR CODE
%----------------------------------------------------------------------------------------

% Renders a qrcode image (centered, relative to the parentwidth)
% param 1: percent width, from 0 to 1
\newcommand{\cvqrcode}[1] {
	\begin{center}
		\includegraphics[width={#1}\mpwidth]{qrcode}
	\end{center}
}


%============================================================================%
%
%
%
%	DOCUMENT CONTENT
%
%
%
%============================================================================%
\begin{document}
\columnratio{0.31}
\setlength{\columnsep}{2.2em}
\setlength{\columnseprule}{4pt}
\colseprulecolor{lightcol}
\begin{paracol}{2}
\begin{leftcolumn}
%---------------------------------------------------------------------------------------
%	META IMAGE
%----------------------------------------------------------------------------------------
\includegraphics[width=\linewidth]{untitled.png}	%trimming relative to image size

%---------------------------------------------------------------------------------------
%	META SKILLS
%----------------------------------------------------------------------------------------
\cvsection{SKILLS}

\cvskill{C++} {2 yrs} {0.85} \\[-2pt]

\cvskill{GoogleTest} {2 yrs} {0.9} \\[-2pt]

\cvskill{C} {3 yrs} {0.5} \\[-2pt]

\cvskill{CMAKE} {1 yrs} {0.4} \\[-2pt]

\cvskill{GIT} {3+ yrs} {0.80} \\[-2pt]

\cvskill{Linux} {2+ yrs} {0.75} \\[-2pt]

\vfill\null
\cvsection{CONTACT}
	
\icontext{MapMarker}{0}{Gibraltarkroken 4, LGH 1317\\412 79 Gothenburg}{black}\\[6pt]
\icontext{MobilePhone}{0}{+46 73 528 93 11}{black}\\[6pt]
\iconemail{Envelope}{0}{hastmark2001@hotmail.com}{contact@example.com}{black}\\[6pt]
\iconhref{Github}{0}{github.com/Manfred-Hastmark}{https://github.com/jankapunkt}{black}\\[6pt]

\vfill\null
\cvqrcode{0.7}

%---------------------------------------------------------------------------------------
%	EDUCATION
%----------------------------------------------------------------------------------------
\newpage
\cvsection{EDUCATION}

\cvmetaevent
{2020-2023}
{Bachelors, Computer Science and Engineering}
{Chalmers University of Technology}
{In my bachelors I got to learn about basic computer architecture
as well as the basics of algorithms and embedded software. Furthermore,
we had basic courses in linear algebra, statistics, calculus and electronic circuits.}

\cvmetaevent
{2023-2025}
{Masters, High Performance Computing}
{Chalmers University of Technology}
{The main topic of my masters is how heterogenous computing can be used to either increase
energy efficiency or to speedup computer programs. Some noteable courses include an introductory course
to HPC parallel computing which included GPU program with cuda, as well as parallell computing with OMP and MPI.
Another highlight was also an advanced course in real time programming were multiple scheduling algorithms were surveyed
and a basic proof of concept real time music player was implemented on a distributed system of 
three STM32 processors with CAN communication.}

\mbox{} % hotfix to place qrcode on the bottom when there are not other elements
\vfill\null
\cvqrcode{0.7}

\end{leftcolumn}
\begin{rightcolumn}
%---------------------------------------------------------------------------------------
%	TITLE  HEADER
%----------------------------------------------------------------------------------------
\fcolorbox{white}{darkcol}{\begin{minipage}[c][3.5cm][c]{1\mpwidth}
	\begin {center}
		\HUGE{ \textbf{ \textcolor{white}{ \uppercase{ MANFRED HÄSTMARK } } } } \\[-24pt]
		\textcolor{white}{ \rule{0.1\textwidth}{1.25pt} } \\[4pt]
		\large{ \textcolor{white} {High Performance Computing Masters Student at Chalmers} }
	\end {center}
\end{minipage}} \\[14pt]
\vspace{-12pt}

%---------------------------------------------------------------------------------------
%	PROFILE
%----------------------------------------------------------------------------------------
\cvsection{PROFILE}

\cvtext{
High-Performance Computing Master's student at Chalmers with a passion for embedded software development.
Experienced in C++ in both student projects, such as Chalmers Formula Student,
and in professional environments at Ericsson. Some highlight projects which I have developed
are a BMS-system, integration of unit-testing in Chalmers Formula Student, and a emigration from
singleton-patterns to dependency injection at Ericsson.
}

%---------------------------------------------------------------------------------------
%	WORK EXPERIENCE
%----------------------------------------------------------------------------------------
\cvsection{WORK EXPERIENCE}

\cvevent
	{JUNE 23 - NOW}
	{Embedded C++ developer at Ericsson}
	{Research and Development}
	{I did an internship during the summer of 2023 and has since worked parttime in parallell with school. 
    The work I have conducted can be summarized as restructuring software patterns in the startup 
    procedure software for the base stations.}
	{\cvlist{
		\item Changing design pattern from singleton -> dependency injection
		\item Uplifting GoogleTest version
		\item General code cleanup
	}}
	{\cvlist {
		\item C++
		\item GoogleTest
		\item Linux
		\item Gerrit
	}}
	{}

\cvevent
	{SEP 24 - DEC 24}
	{Manager Chalmers Formula Student}
	{Embedded/LV sub-group}
	{The purpose of the manager role in Chalmers Formula Student is to onboard the new team
    and set major concepts, and give guidance for next years car design. The manager group along
    with the managers for all subsystem of the car consisted of 15 persons and were selected from the previous
    years project engineers by a faculty advisor.
    }
	{\cvlist{
		\item Recruited and held interviews
		\item Co-created concept for LV hardware
		\item Implemented CAN interface for inverter client
		\item Helped plan for and guide new team
	}}
	{}
	{}

\vfill\null
\cvevent
	{SEP 23 - SEP 24}
	{Engineer Chalmers Formula Student}
	{Embedded Software Engineer}
	{I was Co-responsible for the embedded software together with one other person.
    Together we did a major refactoring of the entire code-base in order to create
    a more modular and testable system. Generally, a lot of the design principles that I had
    seen and worked with at Ericsson were implemented along with some functionality changes to accomodate
    new hardware.}
	{\cvlist{
		\item New BMS software
		\item Unit-test integration with GoogleTest
		\item Restructured and modularized project structure
		\item Integrated new workflow inspired from V-model
		\item Many other smaller projects...
	}}
	{\cvlist {
		\item C++
		\item GoogleTest
		\item Gitlab
		\item Foxglove
		\item Basic electronics hardware knowledge
	}}
	{}

% hotfixes to create fake-space to ensure the whole height is used
\mbox{}
\vfill
\mbox{}
\vfill
\mbox{}
\vfill
\mbox{}
\end{rightcolumn}
\end{paracol}
\end{document}

